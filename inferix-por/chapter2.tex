\chapter{Scene watermarking}
\label{ch:scene_watermarking}
% \chapterprecishere{Nani gigantum humeris insidentes}
\epigraph{Nanos gigantum humeris insidentes}{Bernard of Chartres}

Before being sent back to the users, the output images of \emph{workers} must be checked if they are actually the results of the corresponding submitted graphics scene. By architectural design, Inferix gains no control whatsoever on \emph{workers} (see the discussion in~\autoref{sec:output_verification}), any software installed on \emph{workers} always suffer from risks of being completely analyzed, reversed and maliciously modified. In the long run, one cannot make any assumption about the security on the \emph{worker} side, even worse \emph{workers} should be considered adversaries who exhaustively try to bypass any security enforcement applied on them.
% , one cannot suppose any security on 
% the graphics rendering programs installed on workers are secured.

A public-key cryptography approach is using a scheme of \emph{fully homomorphic encryption} (FHE)~\cite{Gentry2009}. The graphics scene is encrypted first by a private key before sending to \emph{workers}. Given the corresponding public key, the homomorphic encryption software performs the graphics rendering on the encrypted scene without needing to decrypt it. Finally, the encrypted rendered results are returned and decrypted at the user's side using the private key. The advantage of FHE is that the \emph{workers}, even being able to modify the FHE softwares on their side, cannot interfere the FHE graphics rendering processes without being detected. Unfortunately, this approach is impractical since all state-of-the-art implementations will make the performance of the homomorphic encryption graphics rendering become unacceptable~\cite{9910347}.

To handle this problem, we use a watermarking~\cite{Cox1997,Cox1999} method called \emph{Active Noise Generation and Verification} (ANGV) which is a simpler variant of \emph{proof of ownership}~\cite{Wu1998,Yeung1997} schemes. \emph{First}, the \emph{verifier} can quickly judge that the rendered digital content is an authentic result of a rendering process whose input is the submitted graphics scene. \emph{Second}, though ANGV needs to modify the initial submitted scene, as a sequence the rendered output will be distorted also, the distortion is regulated to below the human perception capability, hence there is no quality degradation. \emph{Third}, the embedded noise is robust under rendering enhancements (e.g. anti-aliasing) and post-processing operations. \emph{Finally}, there is no need to use a special graphics rendering software on \emph{workers} as in the case of FHE, the performance of graphics rendering is not affected.
% verification will be proceeded on \emph{verifiers}

% the verification is proceeded on \emph{verifiers}

\section[Active noise generation and verification]{Active noise generation and verification}
% As discussed above (see also~\ref{sec:output_verification}), to check the validity of the rendered images output from \emph{workers}, special watermarks will be embedded into these images.
% It is important to note that we cannot insert watermarks into the images as in the traditional digital watermarking methods~\cite{Cox2008} since the images are output of \emph{workers}, they are not under our control.
% Since images are output of \emph{workers} which are not under control, watermarks cannot be embedded directly into 
A user submits some graphics work to a \emph{manager} (see~\autoref{sec:rendering_network}), this work consists of several graphics scenes; each contains information about graphical objects, the camera, light sources and materials~\cite{Blender}. The photorealistic rendering is a sophisticated computation process that calculate light properties at surfaces of all visible objects, results in 3D rendered images of the scene~\cite{Hughes2014}.

In traditional digital watermarking schemes~\cite{Cox1997,Cox1999}, to check the authenticity, an invisible watermark~\cite{Yeung1997,Craver1997,Wu1998} will be embedded into the image needed to be protected. A detector tries to extract the watermark from a tested image then compares the extracted watermark with the original embedded one, if the comparison is passed then that the tested image is concluded as being authentic.

\paragraph[Approach]{Approach}\label{par:inferix_approach}
In the context of Inferix's rendering network, the \emph{manager} has access to the image only after the graphics scene has been rendered by some \emph{worker}. It is nonsense to embed watermark into the image at this point since the watermarking cannot help to detect any malicious manipulations which may happen before that, namely in the rendering process. Our approach is to embed watermarks into the graphics scene submitted by users before sending it to \emph{workers}.

% the \emph{manager} has no access to the rendered images

% To check whether an image output from a \emph{worker} is actually an authentic result of a genuine graphics rendering process whose input is a given scene, invisible watermarks~\cite{Craver1997} will be embedded into the rendered images. A \emph{verifier} which knows the watermarks  It might be worth noting that, different from the context of traditional digital watermarking schemes~\cite{Cox2008}, these images are not under our control, they are instead output of \emph{workers}. Hence, it is not possible to embed directly watermarks into images, our approach is to embed watermarks into the graphics scene submitted by users before sending it to \emph{workers}.

% the initial digital content in this context is now the output of the malicious side, so freely susceptible to malicious intervention.

\subsection[Hight level description]{High level description}
The \emph{Active Noise Generation and Verification} (ANGV) consists of two main procedures as described below.

\paragraph[Noise generation]{Noise generation}
In practice, a graphics scene may contain multiple frames, each job of this scene contains some range of frames to be rendered, consequently each worker may render only a subset of these frames. For the simplification purpose, we assume in this section that a scene has only one frame, so the output image is determined uniquely by the scene.
\begin{figure}[h]
    \centering
    \includegraphics[width=0.8\textwidth]{noise_generation.png}
    \caption[Noise generation]{Noise generation}
    \label{fig:noise_generation}
\end{figure}
Mathematically, we denote the graphics rendering process by a function $\mathcal{R}$, for each input scene $\mathcal{S}$, the result of the rendering is an image:
\begin{equation*}
    I = \mathcal{R} \left( S \right)
\end{equation*}
% Let $\mathcal{R}$ and $S$ denote respectively the rendering function and the scene, the rendered image $I$ without watermark will be:
% \begin{equation*}
%     I = \mathcal{R} \left( S \right)
% \end{equation*}
It is important to note that $I$ is actually never computed, neither by the \emph{manager} in the watermark insertion procedure nor by \emph{workers} in rendering processes, the equation above represents only an equality.

Similar with invisible watermark schemes~\cite{Yeung1997,Craver1997,Wu1998}, a noise $W$ is a sequence of atomic watermarks:
\begin{equation*}
    W = \left\{ w_1,\dots,w_n \right\}
\end{equation*}
where $w_i$ is chosen from some probability distribution and $w_i$ may also depend on $S$ (see~\autoref{par:robustness}). Using $W$ and $S$, we next generate a verification key:
\begin{equation*}
    K_{\mathtt{verif}} \left(W,S\right) = \left\{ k_1,\dots,k_n \right\}
\end{equation*}
that will be used later for the watermark verification procedure.

The noise $W$ is not embedded into the image $I$ but into the scene $S$ (see~\autoref{fig:noise_generation}). We use an encoding function $\mathcal{E}$ to create a watermarked scene:
\begin{equation*}
    \hat{S} = \mathcal{E} \left(S, W\right)
\end{equation*}
Finally $\hat{S}$ is sent to \emph{workers} for rendering, that results in a watermarked image:
\begin{equation*}
    \hat{I} = \mathcal{R} \thinspace (\hat{S})
\end{equation*}
In case of being accepted, the rendered image sent back to the original user (i.e. the owner of the graphics scene $\mathcal{S}$) is $\hat{I}$ but not $I$. The encoding function $\mathcal{E}$ and the watermark $W$ are designed so that the distortion of $\hat{I}$ against $I$ is under human perception capability, then $\hat{I}$ can be authentically used as a result of the graphics rendering.

\paragraph[Noise verification]{Noise verification}
\begin{figure}[h]
    \centering
    \includegraphics[width=0.8\textwidth]{noise_verification.png}
    \caption[Noise verification]{Noise verification}
    \label{fig:noise_verification}
\end{figure}
Given an image $J$ and a verification key $K_{\mathtt{repr}}$, we first try to recover a watermark $\hat{W}$ from $\hat{I}$ using a decoding function $\mathcal{D}$:
\begin{equation*}
    \hat{W} = \mathcal{D} \thinspace (J, K_{\mathtt{repr}})
\end{equation*}
Next $\hat{W}$ is compared against $W$, if the difference is above some threshold $T$
\begin{equation*}
    \lVert \hat{W} - W \rVert \geq T
\end{equation*}
then $J$ will be accepted otherwise rejected.

% \paragraph[Rendering result]{Rendering result}
% In case of being accepted, the rendered image sent back to users is $\hat{I} = \mathcal{R}\thinspace(\hat{S})$ but not $I = \mathcal{R}\left(S\right)$. The encoding function $\mathcal{E}$ and the watermark $W$ are designed so that the distortion of $\hat{I}$ against $I$ is under human perception capability, then $\hat{I}$ can be authentically used as a result of the graphics rendering.

\paragraph[Performance]{Performance}
The functions $\mathcal{E}$ and $\mathcal{D}$ are designed so that their computation resource consumption is negligible in comparison with the graphics rendering function $\mathcal{R}$; and the watermark verification does not require $I$, only $\hat{I}$ is rendered. That means the only overhead caused by ANGV is by $\mathcal{E}$ and $\mathcal{D}$, hence there is no significant effect on the performance of the graphics rendering.
% (there is no need to render this image for comparing with $I'$)

\subsection[Deployment]{Deployment}
To apply ANGV into the rendering network (see~\autoref{fig:rendering_service})
 

\subsection[Implementation]{Implementation}
In this section, we present in detail the current Inferix's implementation of watermark insertion and verification.

% \subsubsection[Insertion]{Insertion}

\paragraph[Structure of watermark]{Structure of watermark}\label{par:structure_of_watermark}
As described above, a watermark $W$ consists of a sequence of $n$ atomic watermarks, each watermark $w_i$ is a rectangle image of special pattern (see~\autoref{fig:atomic_patterns}) of noise chosen from a normal distribution $w_i \sim \mathcal{N}(0, \sigma_i^2)$.
\begin{figure}[h]
    \centering
    \begin{subfigure}[b]{0.25\textwidth}
        \includegraphics[width=\textwidth]{w_inferix_0.png}
    \end{subfigure}
    \qquad
    \begin{subfigure}[b]{0.25\textwidth}
        \includegraphics[width=\textwidth]{w_inferix_1.png}
    \end{subfigure}
    \caption[Some atomic watermark patterns]{Some atomic watermark patterns}
    \label{fig:atomic_patterns}
\end{figure}
% \begin{figure}[h]
%     \centering
%     \includegraphics[scale=0.2]{w_inferix.png}
%     \caption{An atomic watermark}
%     \label{fig:atomic_watermark}
% \end{figure}
The number $n$ is one of the factors decides the robustness of watermark, the higher this number the lower the false positive of watermark verification. Experimentally, $W$ contains about $8$ to $10$ atomic watermarks.

\paragraph[Watermark insertion]{Watermark insertion}
Given some graphics scene $S$, we embed the watermark $W$ into $S$ by wrapping each atomic watermark as a graphics objects, then insert these objects into $S$ so that they contribute to the result rendered image (i.e. they distort this image).

By the nature of the graphics rendering, an object in the scene will not be visible if there is no light scattered from the surface of the object to the camera. This may be caused by several reasons: the object is not located in the frustum of the camera, is hidden by some other objects, or the material of the object is transparent, etc. The watermark insertion must satisfy first several constraints:
\begin{itemize}
    \item there are no collisions between watermark objects,
    \item all watermark objects are located completely in the camera frustum,
    \item no watermark object is hidden by another object (including both watermark objects and existing objects of the scene).
\end{itemize}

\paragraph[Representative key generation]{Representative key generation}
An atomic watermark $w_i$ contributes to the image $\hat{I}$ as a rectangular region at the location $k_i$ (for that, the rotation vector of the graphics object $w_i$ is kept to be equal with the one of the camera in $S$). The location $k_i$ on $\hat{I}$ can be precisely calculated without rendering:
\begin{equation*}
    k_i = \left(x^{\mathtt{ul}}_i, y^{\mathtt{ul}}_{i},x^{\mathtt{lr}}_i, y^{\mathtt{lr}}_{i}\right)
\end{equation*}
where $\left(x^{\mathtt{ul}}_i, y^{\mathtt{ul}}_{i}\right)$ (resp. $x^{\mathtt{lr}}_i, y^{\mathtt{lr}}_{i}$) are the upper left (resp. lower right) coordinates of the region. So the insertion of $W$ into $S$ generates a key
\begin{equation*}
    K_{\mathtt{repr}} = \left\{k_1,\dots,k_n \right\}
\end{equation*}

\paragraph[Distortion constraint]{Distortion constraint}
It might be worth noting that $k_i$ represents the distortion caused by the atomic watermark $w_i$ to the rendered image, normally $k_i$ is much smaller than the original size of $w_i$. On the one hand $k_i$ must not be negligible otherwise the watermark verification is not possible. On the other hand, $k_i$ must be kept under the human perception capability. Experimentally we keep a trade-off:
\begin{align}\label{eq:side_constraint}
\begin{split}
    2 \leq x^{\mathtt{lr}}_{i} - x^{\mathtt{ul}}_{i} \leq 3 \\
    2 \leq y^{\mathtt{lr}}_{i} - y^{\mathtt{ul}}_{i} \leq 3
\end{split}
\end{align}
for all $1 \leq i \leq n$ (see~\autoref{fig:coca_cola} and~\autoref{fig:tea_mug}).
\begin{figure}[ht]
    \centering
    \makebox[\textwidth][c]{%
    \begin{subfigure}[t]{0.4\textwidth}
        \centering
        \includegraphics[width=\textwidth]{coca_cola.png}
        \label{subfig:coca_cola}
        \caption{Original}
    \end{subfigure}
    % \quad
    \begin{subfigure}[t]{0.4\textwidth}
        \centering
        \includegraphics[width=\textwidth]{coca_cola_wm_ko.png}
        \label{subfig:coca_cola_wm_ko}
        \caption{Watermarked with atomic watermark size $= 5$}
    \end{subfigure}
    % \vfill
    % \qquad
    \begin{subfigure}[t]{0.4\textwidth}
        \centering
        \includegraphics[width=\textwidth]{coca_cola_wm_ok.png}
        \label{subfig:coca_cola_wm_ok}
        \caption{Atomic watermark size $= 2$}
    \end{subfigure}}
    \caption[Scene watermarked with different sizes of atomic watermarks]{Scene watermarked with different size of atomic watermarks: the distortion is observable when the width and the height of an atomic watermark are about $5$, but unobservable when these lengths are $2$.}
    \label{fig:coca_cola}
\end{figure}

\begin{figure}[ht]
    \centering
    \makebox[\textwidth][c]{%
    \begin{subfigure}[t]{0.4\textwidth}
        \centering
        \includegraphics[width=\textwidth]{tea_mug.png}
        \label{subfig:tea_mug}
        \caption{Original}
    \end{subfigure}
    \begin{subfigure}[t]{0.4\textwidth}
        \centering
        \includegraphics[width=\textwidth]{tea_mug_wm_ko.png}
        \label{subfig:tea_mug_wm_ko}
        \caption{Atomic watermark size $= 7$}
    \end{subfigure}
    \begin{subfigure}[t]{0.4\textwidth}
        \centering
        \includegraphics[width=\textwidth]{tea_mug_wm_ok.png}
        \label{subfig:tea_mug_wm_ok}
        \caption{Atomic watermark size $= 3$}
    \end{subfigure}}
    \caption[Scene watermarked with different sizes of atomic watermarks]{Another scene watermarked with different sizes of atomic watermarks}
    \label{fig:tea_mug}
\end{figure}

\paragraph[Watermark verification]{Watermark verification}
The verification procedure needs only the representative key $K_\mathtt{repr}$ and the rendered image $\hat{I}$. It is not always possible to restore the atomic watermark $w_i \in W$ since most information of $w_i$ is lost after the graphics rendering process (note that $k_i$ is much smaller than the size of $w_i$). However, it is neither necessary to restore $w_i$, instead to verify the existence of $W$, checking the existence of $w_i$ on the image region determined by $k_i$ on $\hat{I}$ for $1 \leq i \leq n$ is sufficient.

For each region of coordinates $k_i = \left(x^{\mathtt{ul}}_i, y^{\mathtt{ul}}_{i},x^{\mathtt{lr}}_i, y^{\mathtt{lr}}_{i}\right)$, we pick a bound region of coordinates:
\begin{equation*}
    b_i =  \left(x^{\mathtt{ul}}_i - \delta^{x}_{i}, y^{\mathtt{ul}}_{i} - \delta^{y}_i ,x^{\mathtt{lr}}_i + \delta^{x}_{i}, y^{\mathtt{lr}}_{i} + \delta^{y}_{i}\right)
\end{equation*}
where $\delta^{x}_{i}$ and $\delta^{y}_{i}$ are the width and the height of $k_i$ (see~\autoref{subfig:coca_cola_pickup}):
\begin{equation*}
    \delta^{x}_{i} = x^{\mathtt{lr}}_i - x^{\mathtt{ul}}_i \qquad \delta^{y}_{i} = y^{\mathtt{lr}}_i - y^{\mathtt{ul}}_i \\
\end{equation*}
To check the contribution of the atomic watermark at the region of coordinates $k_i$, we apply an edge detection filter (see~\autoref{subfig:coca_cola_pickup_filtered}) on the bound region, then compare the mean $m_b$ of the bound region against the mean $m_k$ of the atomic watermark located inside. Experimentally, we will check if
\begin{equation}\label{eq:edge_difference}
    m_k - m_b \geq 5
\end{equation}
to validate the existence of the atomic watermark. This check is proceeded for all $n$ atomic watermarks, if all of them are validated then the rendered image $\hat{I}$ is accepted otherwise rejected.
\begin{figure}[ht]
    \centering
    \begin{subfigure}[t]{0.45\textwidth}
        \includegraphics[width=\textwidth]{coca_cola_bound.png}
        \caption{Pickup bound region of size $9\times9$ pixels, the atomic watermark region of size $3\times3$ located at the center of the bound (numbers on each pixel are RGB color values)}
        \label{subfig:coca_cola_pickup}
    \end{subfigure}
    \qquad
    \begin{subfigure}[t]{0.45\textwidth}
        \includegraphics[width=\textwidth]{coca_cola_laplace.png}
        \caption{Pickup bound region after an edge detection filter}
        \label{subfig:coca_cola_pickup_filtered}
    \end{subfigure}
    \caption{Watermark verification}
    \label{fig:coca_cola_bound}
\end{figure}

\paragraph[Robustness]{Robustness}\label{par:robustness}
We now discuss several details about the contribution of the watermark on the rendered image. There are two related constraints:
\begin{itemize}
    \item the side trade-off in~\autoref{eq:side_constraint} is used to keep the rendering distortion under the human perception capability, and
    \item the mean difference in~\autoref{eq:edge_difference} is used to validate the existence of an atomic watermark
\end{itemize}
which are unfortunately disproportional: if the side is too small

\subsection[Performance]{Performance}

\section[Threat analysis]{Threat analysis}